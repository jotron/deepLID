\section{Einleitung}
Sprachoberflächen sind der aktuelle Megatrend in der Tech-Branche. Ein Algorithmus, der natürliche Sprache zuverlässig versteht, ist längst kein Science-Fiction mehr. High-Tech-Firmen rund um den Globus investieren Milliarden in die Entwicklung solcher Produkte. Genau wie das iPhone mit dem Touchscreen eine Revolution der Interaktion Mensch-Maschine lancierte, erhofft man sich mit Spracherkennung den nächsten Durchbruch. Sprachgesteuerte Programme sollen weitaus intuitiver zu bedienen sein als Texteingaben. In Zukunft ist die lästige Uhrzeitabfrage-Bewegung zum Handy wahrscheinlich Geschichte. Wir werden bequem danach fragen können.
\\ \\
Moderne Spracherkennung-Systeme sind auf einzelne Sprachen spezialisiert. Aus diesem Grund muss ein sprachunabhängiges System, in erster Linie die verwendete Sprache identifizieren, um anschliessend, den passende Sprachassistenten anzuwenden. Der sprachabhängige Sprachassistent kann dann zum Beispiel die Grammatik der erkannten Sprache zur Hilfe nehmen, um nur grammatikalisch korrekte Sätze zu erkennen. In dieser Arbeit geht es darum den Prozess der Sprachidentifikation zu implementieren:  Aufnahmen mündlicher Äusserungen sollen der korrekten Sprache zugeordnet werden. 
\\ 
Die Idee zu diesem Thema entstand spontan. Für mich war aber seit Anfang an klar, dass ich meine Maturaarbeit im Fach Informatik schreiben wollte. Ich nahm bereits an der Informatikolympiade und dem Freifach \textit{Begabtenförderung Informatik} teil. Mich weiter in diesen Bereich zu vertiefen, war eine logische Konsequenz. Überdies besitzen die meisten Informatik-Projekte den Vorteil äusserst ressourcenarm zu sein. Ein Computer mit Internet-Anschluss reicht, um ein innovatives Produkt zu entwickeln, dass die Welt verändern kann. Das nötige Wissen lässt sich ohne lange Ausbildung im Internet finden. Entsprechend erlaubt Informatik Jugendlichen, bereits Teil der Wirtschaft zu sein. Kein 14-Jähriger kann, ohne teures Labor Medikamente entwickeln, hingegen für die Idee \textit{Blockchain} hätte theoretisch jeder die Mittel gehabt.
\\ \\
Für die Sprachidentifikation verwende ich in dieser Arbeit die Methode \textit{Künstliche Intelligenz} (KI). KI ist ein prominentes junges Teilgebiet der Informatik. In den letzten Jahren wurden enorme Fortschritte gemacht, was dazu führt, dass die Wirtschaft der Forschung weit hinterherhinkt. Dank fortgeschrittener Computer-Hardware und neuen Algorithmen können Maschinen heute Aufgaben lösen, die vor zehn Jahren unmöglich erschienen.
Mittlerweile sind zum Beispiel sogar selbst-fahrende Autos nur noch eine Frage der Zeit. Dem Traum der Menschen, eine ihr in Intelligenz ebenbürtige Maschine zu entwerfen, kommt man langsam aber stetig näher.
\\
Die klassischen Projekte in KI-Arbeiten sind oft Spiele. Es wird dem Computer beigebracht besser zu spielen als jeder Mensch. So ist es zum Beispiel beim Schachcomputer
\textit{Deep blue}\parencite{deepblue} , dass 1996 Garry Kasparov schlug, oder dem Go-Programm \textit{AlphaGo}\parencite{alphago}.
Solche Projekte bieten zwar gute Forschungsobjekte, sind aber an sich keine Anwendungsfelder. Ich wollte in meiner Arbeit etwas programmieren, das tatsächlich einen realen Nutzen
besitzt.
Sprachidentifikation erfüllt das Kriterium. Sprachidentfikation kann nebst für die Spracherkennung, unter anderem im Anrufcenter angewendet werden. Ein Computer erkennt die Sprache des Kunden und leitet den Anruf an den passenden Mitarbeiter weiter.
\\ \\
Klassische Methoden für die Sprachidentifikation beruhen stark auf Expertenwissen. Analytisch entwickelte, handprogrammierte Verfahren erreichen mit wenig maschinellem Lernen sehr gute Leistungen. In dieser Arbeit wird ein anderer Ansatz gewählt. Der Computer soll möglichst viel selbst erlernen. Aus diesem Grund verwende ich die Technologie \textit{Deep Learning}. 
\\ 
Das Ziel dieser Arbeit lässt sich in zwei Teile gliedern. Zuerst soll KI bzw. Deep Learning beschrieben und soweit wie möglich ein Verständnis dafür geschaffen werden.
Anschliessend wird die Technologie auf das Problem Sprachidentifikation angewendet. Als zu identifizierende Sprachen gelten Französisch, Englisch und Deutsch. Es
werden verschiedene Ansätze probiert und verglichen. Das Produkt soll eine möglichst grosse Fehlerfreiheit besitzen. Um das Produkt zu demonstrieren, wird ein Web-Interface
entwickelt, das einem ermöglicht, das Produkt selber zu testen.
\\ \\
Um die Realisierbarkeit dieses Vorhabens zu beurteilen, hatte ich im Vorfeld der Arbeit, bereits das Buch  \textit{Neuronale Netze selbst programmieren}\parencite{neuronale_netze} aus dem Info-Z gelesen. Dies ermöglichte mir erst, die Fragestellung genauer einzuschränken und den nötigen Aufwand dafür abzuschätzen. Die enthaltenen Informationen waren aber noch lange nicht ausreichend, um mit dem Programmieren zu beginnen. Ausschlaggebend war erst später das Buch \textit{Deep Learning with Python}\parencite{chollet}, das als theoretische Grundlage dient. 
\\
Parallel dazu wurde begonnen, das Web-Interface zu entwickeln. Das nötige Know-How besass ich bereits grösstenteils als Vorwissen. Das dann realisierte Interface erlaubte mir, das Produkt in der Entwicklungsphase live zu testen. 
\\
Der nächste Schritt war endlich das Programmieren am Produkt. In dieser Phase gab das Paper \textit{Practical Applications of Multimedia Retrieval. Language Identification in Audio
Files}\parencite{iLID} die Richtung vor. 
Um die Unterschiede zwischen den Sprachen zu erlernen, benötigt mein Programm viele Trainings-Daten. Die Beschaffung dieser grossen Menge Daten und das effiziente Umgehen damit, sind ein nicht zu unterschätzender Aufwand. 
\\
Die letzte Phase, das Entwickeln der künstlichen Intelligenz, war am spannendsten. Die Komplexität des Themas erlaubte allerdings oft nur oberflächlich, ein Verständnis zu entwickeln, zumal die mathematischen Grundlagen auf Universitäts- Stufe fehlten. 
\\ \\
Im Kapitel \textbf{2 Deep Learning} beschreibe ich kurz die theoretischen Grundlagen zur künstlichen Intelligenz. Das Kapitel bildet das Herz des theoretischen Teils. Das Kapitel \textbf{3
Daten} befasst sich mit den Quellen der Daten, den Methoden zur Datenbeschaffung und den Algorithmen zur Datenbearbeitung. Anschliessend werden in Kapitel \textbf{4 Modelle} die entwickelten KI-Modelle präsentiert. Kapitel \textbf{5 Umsetzung und Resultate} erläutert die Implementierung des Systems und zeigt die Auswertung der Modelle. In \textbf{6 Diskussion} werden die
Ergebnisse in Zusammenhang gestellt und Kapitel \textbf{7 Anhang} gibt Interessierten einen Einblick in einen Teil des Codes. Der Vollständige Programmiercode der Arbeit ist online hier verfügbar: \url{https://github.com/jotron/deepLID}
