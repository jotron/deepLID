\section{Einleitung}
\subsection{Themenwahl}
Für mich war seit Anfang an klar, dass ich meine Maturaarbeit im Fach Informatik beschreiten will. Ich interessierte mich schon lange für das Fach, besuchte das Freifach
Begabtenförderung-Informatik  und war darum auch nicht ganz ohne Vorwissen. Informatik erlaubt wie kein anderes Fach ohne grössere Ressourcen ein konkurrenzfähiges Produkt zu 
gestalten. Würde ich eine Informatik-Startup lancieren, bräuchte ich dafür nur meinen Computer und einen Internet-Anschluss, nichts anderes. Das nötige Wissen findet man ohne lange
Ausbildung im Internet. Auf diese Weise ermöglicht Informatik eine wahrhaftige Produktivität der Jugend.
\\ \\ 
Das Teilgebiet auf die Künstliche Intelligenz zu fixieren nahm mir längere Zeit. KI hat aber mehrere Vorteile gegenüber anderen Themen. 

Erstens ist es extrem aktuell. In den letzten Jahren wurden enorme Fortschritte gemacht was dazu führt, dass die Industrie der Forschung weit hinterher ist. Dank fortgeschrittener
Computer-Hardware und neuen Algorithmen können Maschinen heute Aufgaben lösen die vor zehn Jahren unmöglich erschienen. Diese Aufblühende Phase wird generell KI-Sommer
genannt.

Zweitens ist KI für mich faszinierend. Künstliche Intelligenz fasziniert den Menschen bereits seit langer Zeit. Das Prinzip, einer Maschine Teile unserer kognitiven Fähigkeiten
beizubringen ist für viele der nächste Schritt der menschlichen Evolution.
Zuletzt überlege ich mir mein Studium in Richtung Informatik/Ki zu absolvieren. Diese Arbeit ist für mich persönlich also ein Test, ob dies das richtige für mich ist.

Die klassischen Projekte in KI-Arbeiten sind oft Spiele. Es wird dem Computer beigebracht besser zu spielen als jeder Mensch. So ist es zum Beispiel beim Schachcomputer
\textit{Deep blue}\parencite{deepblue} oder dem Go-Programm \textit{AlphaGo}\parencite{alphago}.
Solche Projekte bieten zwar gute Forschungsprojekte sind aber an sich keine Anwendungsfelder. Ich wollte in meiner Arbeit etwas programmieren, das tatsächlich einen realen Nutzen
hat.
Schlussendlich bin ich auf Sprachidentifikation gestossen. Mit zwei Muttersprachen bin ich perfekt geeignet, mein Projekt zu testen und Sprachmaterial lies sich genügend im Internet
finden. Das war Ausschlaggebend.

\subsection{Ziel der Arbeit}
Bei Sprachidentifikation geht es darum, anhand von Audio-Aufnahmen die gesprochene Sprache zu bestimmen. Eine fortgeschrittene Anwendung ist Spracherkennung: Moderne
Computer und Mobiltelefone interagieren zunehmlich mit uns über unsere Stimme. Dank schneller Sprachidentifikation kann das Gerät die Grammatik der Sprache dabei zur Hilfe
nehmen.
Eine weitere Anwendung sind zum Beispiel Support-Anrufe. Eine Computer erkennt die Sprache und leitet den Anruf an den passenden Mitarbeiter weiter.

Klassische Methoden für die Sprachidentifikation beruhen stark auf Expertenwissen. Die Signale werden mehrfach mit komplexen Operationen bearbeitet um dann mit nicht weniger
einfacheren Algorithmen ausgewertet zu werden.
In dieser Arbeit wird ein anderer Ansatz gewählt. Der Computer soll möglichst viel selbst erlernen. Dafür beschränke ich mich auf die Technologie \textit{Deep Learning}.
\\ \\ 
Das Ziel dieser Arbeit lässt sich in zwei Teile gliedern. Zuerst soll KI bzw. Deep Learning beschrieben werden und soweit wie möglich ein Verständnis dafür geschaffen werden.
Anschliessend wird die Technologie auf das Problem Sprachidentifikation angewendet. Als zu identifizierende Sprachen beschränke ich mich auf Französisch, Englisch, und Deutsch. Es
werden verschiedene Ansätze probiert und verglichen. Das Produkt soll eine möglichst grosse Fehlerfreiheit besitzen. Um das Produkt zu demonstrieren, wird ein Web-Interface
entwickelt das einem ermöglicht das Produkt selber zu testen.

\subsection{Vorgehen}
Ich hatte a priori der Arbeit bereits das \textit{Neuronale Netze selbst programmieren}\parencite{neuronale_netze} aus dem Info-Z gelesen. Dies ermöglichte mir erst meine
Fragestellung genau einzuschränken und den nötigen Aufwand abzuschätzen.
Die enthaltenen Informationen waren aber noch nicht genügend um mit dem Produkt anzufangen. Deshalb recherchierte ich noch eine Zeit weiter. Besonders hilfreich war das Buch
\textit{Deep learning with Python}\parencite{chollet} was als meine theoretische Grundlage diente.

Parallel fing ich an das Web-Interface zu entwickeln. Das nötige Know-How hatte ich bereits grösstenteils als Vorwissen. Das früh realisierte Interface erlaubte mir mein Produkt in der
Entwicklungsphase live zu testen bzw. zu testen lassen.
\\ \\ 
Der nächste Schritt war endlich das programmieren am Produkt. In dieser Phase gab mir das Paper \textit{Practical Applications of Multimedia Retrieval. Language Identification in Audio
Files}\parencite{iLID} die richtige Richtung vor. 

Um die Unterschiede zwischen den Sprachen zu erlernen braucht mein Programm viele Trainings-Daten. Die Beschaffung dieser grossen Menge Daten und das effiziente umgehen
damit, kostete mich mehr Zeit als eingeschätzt und gab mir schlaflose Nächte.

Die Letzte Phase, das entwickeln der künstlichen Intelligenz, war am spannendsten. Die Komplexität des Themas erlaubte mir allerdings oft nur oberflächlich ein Verständnis zu
entwickeln weil mir die mathematischen Grundlagen auf Universitäts-Stufe fehlten.

\subsection{Aufbau}
Im Kapitel \textbf{2 Deep Learning} werden die Technologien der künstlichen Intelligenz vorgestellt. Das Kapitel bildet das Herz des instruktiven Teils. Das nächste Kapitel \textbf{3
Daten} befasst sich mit den Quellen der Daten, den Methoden zur Datenbeschaffung und zur Datenbearbeitung. Anschliessend wird in Kapitel \textbf{4 Web Interface} das Interface und
die verwendeten Frameworks vorgestellt. Kapitel \textbf{5 Resultate} präsentiert die entwickelten Modelle und vergleicht ihre Fehlerfreiheit. In \textbf{6 Diskussion} werden die
Ergebnisse in Zusammenhang gestellt und Kapitel \textbf{7 Code} erlaubt Interessierten einen Einblick in den nötigen Code.
